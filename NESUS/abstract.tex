\documentclass[runningheads,a4paper]{llncs}
\usepackage{amssymb}
\usepackage[T1]{fontenc}
\usepackage[utf8]{inputenc}
\usepackage{graphicx}
\usepackage{wrapfig}
\usepackage{verbatim}
\usepackage{caption}
\usepackage{hyperref}

% #rubber: set arguments --shell-escape
\begin{document}


\title{Roofline Model applied to NUMA and Heterogeneous Memories}
\titlerunning{Roofline Model for Heterogeneous and NUMA Memories}
\author{Nicolas Denoyelle \and Aleksandar Ilic}
\institute{Inria - France -- INESC-ID -- Portugal\\
  Nicolas.Denoyelle@inria.fr, ilic@sips.inesc-id.pt}

\maketitle

\begin{abstract}

The ever growing complexity of high performance computing systems imposes significant challenges to exploit as much as
possible their computational and communication resources.
Recently, the Cache-aware Roofline Model has has gained popularity due to its simplicity modeling multi-cores with complex memory
hierarchy, characterizing applications' bottlenecks, and quantifying achieved or remaining improvements.
In this short paper we push this model a step further to model NUMA and heterogeneous memories with a handy tool, and spot data
locality bottlenecks on such systems.

\end{abstract}

\end{document}
