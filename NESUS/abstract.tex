\documentclass[twoside,twocolumn,8pt]{extarticle}


% ------
% Fonts and typesetting settings
\usepackage[sc]{mathpazo}
\usepackage[T1]{fontenc}
\linespread{1.05} % Palatino needs more space between lines
\usepackage{microtype}


% ------
% Page layout
\usepackage[hmarginratio=1:1,top=10mm,columnsep=20pt,left=0.8in, right=0.8in]{geometry}
\usepackage[font=it]{caption}
\usepackage{paralist}
\usepackage{multicol}

% ------
% Lettrines
\usepackage{lettrine}


% ------
% Abstract
\usepackage{abstract}
	\renewcommand{\abstractnamefont}{\normalfont\bfseries}
	\renewcommand{\abstracttextfont}{\normalfont\itshape}


% ------
% Titling (section/subsection)
\usepackage{titlesec}
\renewcommand\thesection{\Roman{section}}
\titleformat{\section}[block]{\large\scshape\centering}{\thesection.}{1em}{}

\usepackage{graphicx}
% ------
% Header/footer


\usepackage{fancyhdr}
\pagestyle{fancy}

\setlength\headheight{90.0pt}
\addtolength{\textheight}{-90.0pt}

\fancypagestyle{firststyle}
{
   	\fancyhead[L]{\includegraphics[height=5.5em]{pictures/nesus.pdf}}
	\fancyhead[C]{}
	\fancyfoot{}
	\fancyhead[R]{\small{Book paper template $\bullet$ October 2016 $\bullet$ Vol. I, No. 1}}
	\fancyfoot[RO,LE]{\thepage}
}
	\fancyhead[R]{}
	\fancyhead[L]{}
	\fancyfoot{}
	\fancyhead[C]{\small{Book paper template $\bullet$ October 2016 $\bullet$ Vol. I, No. 1}}
	\fancyfoot[RO,LE]{\thepage}


% ------
% Clickable URLs (optional)
\usepackage{hyperref}

% ------
% Maketitle metadata
\title{\vspace{-10mm}%
	\fontsize{24pt}{10pt}\selectfont
	\textbf{Modeling Emerging Complex Memory Hierarchies With The Roofline Model}
	}
	
		
\author{%
	\large
	\textsc{Nicolas Denoyelle \and Aleksandar Ilic} \\[2mm]
	\normalsize{	Inria - France -- INESC-ID -- Portugal}\\
	\normalsize{	\href{mailto:Nicolas.Denoyelle@inria.fr}{Nicolas.Denoyelle@inria.fr} \href{mailto:ilic@sips.inesc-id.pt}{ilic@sips.inesc-id.pt}}
	\vspace{-5mm}
	}

\date{}

\providecommand{\keywords}[1]{\textbf{\textit{Keywords}} #1}

%%%%%%%%%%%%%%%%%%%%%%%%
\begin{document}



\twocolumn[
  \begin{@twocolumnfalse}

%\maketitle

\thispagestyle{firststyle}


\begin{abstract}
\noindent The ever growing complexity of high performance computing systems imposes significant challenges to exploit as much as
  possible their computational and communication resources.
  Recently, the Cache-aware Roofline Model has has gained popularity due to its simplicity modeling multi-cores with complex memory
  hierarchy, characterizing applications' bottlenecks, and quantifying achieved or remaining improvements.
  In this short paper we push this model a step further to model NUMA and heterogeneous memories with a handy tool, and spot data
  locality bottlenecks on such systems.
\end{abstract}


\keywords{Roofline Model, heterogeneous memory, NUMA, Cache}

 \hrulefill
\bigskip 


\end{@twocolumnfalse}
]

\end{document}

